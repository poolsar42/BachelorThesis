\documentclass[a4paper,12pt]{article}

\usepackage{cmap}		
\usepackage[utf8]{inputenc}			
\usepackage[english,russian]{babel}
\usepackage{framed}
\usepackage{hyperref}
\usepackage{amsmath}
\usepackage{graphicx}
\usepackage[colorinlistoftodos]{todonotes}
\usepackage{wrapfig}
\usepackage{lipsum}
\usepackage{listings}
\usepackage{color}
\usepackage{indentfirst}
\usepackage{times}
\usepackage{textcomp}


\definecolor{mygray}{rgb}{0.4,0.4,0.4}
\definecolor{mygreen}{rgb}{0,0.8,0.6}
\definecolor{myorange}{rgb}{1.0,0.4,0}

\lstdefinestyle{customc}{
  belowcaptionskip=1\baselineskip,
  breaklines=true,
  frame=L,
  xleftmargin=\parindent,
  language=C,
  showstringspaces=false,
  basicstyle=\footnotesize\ttfamily,
  keywordstyle=\bfseries\color{green!40!black},
  commentstyle=\itshape\color{purple!40!black},
  identifierstyle=\color{blue},
  stringstyle=\color{orange},
  numbers=left,
  numbersep=13pt,
  numberstyle=\small\color{mygray},
}
\lstset{escapechar=@,style=customc}
\linespread{1.6}
\newcommand{\HRule}{\rule{\linewidth}{0.5mm}}

\begin{document}

\begin{titlepage}
\begin{center}

\textsc{\Large Московский Физико-Технический Институт}\\
\textsc{\large (Национальный Исследовательский Университет)}\\[1.5cm]

% Upper part of the page. The '~' is needed because \\
% only works if a paragraph has started.
\includegraphics[width=0.25\textwidth]{img/logo.png}~\\[1cm]

\textsc{\Large Бакалаврская работа}\\[0.5cm]

% Title
\HRule \\[0.4cm]
{ \LARGE \bfseries Влияние вышестоящих регуляторных регионов и промоторов на связь между бактериофагами и их хозяевами \\[0.4cm] }

\HRule \\[1.5cm]

% Author and supervisor
\noindent
\begin{minipage}{0.4\textwidth}
\begin{flushleft} \large
\emph{Студент:}\\
Шарафутдинов Эмиль
\end{flushleft}
\end{minipage}%
\begin{minipage}{0.4\textwidth}
\begin{flushright} \large
\emph{Преподаватель:} \\
 Родригес Валера Франсиско Эдуардо
\end{flushright}
\end{minipage}

\vfill

% Bottom of the page
{\large 24 апреля 2021 г.}

\end{center}
\end{titlepage}

\newpage
\section{Аннотация}

Бактериофаги (фаги) - это вирусы, которые избирательно поражают бактериальные клетки. Такая специфика обусловлена
зависимостью фага от клеточного механизма хозяина, который нужен фагу для репликации и биосинтеза вирусных компонентов.
Пример такой зависимости  можно привести для фага Т4. Во время транскрипции, ранняя конкуренция за РНК-полимеразу хозяина
позволяет вирусу полностью захватить механизм репликации путем сигма-присвоения(sigma appropriation) [1]. Недавно
показали[2], что фаги, которые заражают цианобактерии, кодируют ген фотосинтеза. Причём транскрипция этого гена реагирует
на те же условия окружающей среды, что и ген, который кодируется у хозяина. Из этого можно предположить, что существует
связь между регуляторными элементами фага и регуляторными элементами их хозяина. Это поможет лучше предсказывать
взаимоотношения фаг-хозяин.

\newpage
\tableofcontents



\newpage
\section{Введение}

    В одной статье были собраны результаты разных исследований, касающихся взаимоотношений фага T4 и его хоста – E. Coli.
    В этих работах было показано, как ведёт себя фаг, который внедряется в бактерию.
	У фага Т4 нет собственных РНК-полимераз, поэтому он использует РНК-полимеразы хоста для транскрипции. Но почему РНК
	полимераза хоста должна сразу транскрибировать именно с ДНК Т4? Не должна. Но всё же обнаружили, что некоторая часть
	ДНК фага транскрибируется РНК-полимеразой хоста сразу. Эту часть транскрипции назвали ранней транскрипцией фага. На
	этом слайде можно видеть, почему эта ранняя транскрипция просходит. Здесь обозначены разные мотивы промоторов ДНК
	фага (посередине – ранняя, снизу - промежуточная) и хоста (сверху). Эти мотивы, которые находятся в промоторе,
	называются вышестоящими регуляторными элементами. Как видно, в промоторе ранней транскрипции регуляторные элементы
	между фагом и хостом совпадают, и выше -35 участка, что позволяет РНК полимеразе цепляться за ДНК фага также хорошо,
	как за ДНК хоста.  Всё, что до мотива TGn совпадает у всех промоторов. MotA Box в промоторе промежуточной
	транскрипции нужен для зацепки белка, который способствует промежуточной транскрипции. Вкратце, что там просходит: в
	этот момент с РНК, транскрибировавшейся от фага (в ранней транскрипции), транслируются белки, которые модифицируют
	РНК-полимеразы хозяина таким образом, что далее начинают транскрибироваться только ДНК фага. В том числе в начальной
	транскрипции ещё образуется белок MotA, который позволяет ДНК прикрепиться к модифицированной РНК-полимеразе. После
	чего с этой ДНК начинает транскрибироваться РНК. Этот этап уже называют промежуточной транскрипцией. Этот процесс
	присвоения РНК-полимеразы хозяина называют сигма присвоением. Ещё раз. Проиcходит это так:

	В самом начале транслируются белки, которые глушат РНК-полимеразу
	Вместе с этим транслируются белки, которые позволяют при заглушенной полимиразе транскрибироваться ДНК фага на этой
	полимеразе, они садятся на ДНК фага
	Таким образом фаг полностью захватывает РНК-полимеразу хозяина
	Рассмотри ещё один интересный случай для нас. Ещё в одной статье авторы смотрели, как ведут себя транкриптомы фагов и
	их хозяев (цианобактерий в этом случае) в зависимости от интенстивности света, которым они светили. При маленькой
	интенсивности транскриптомы делились в среднем поровну, но при бОльшей интенсивности среднее число РНК фагов было
	намного больше, чем у цианобактерий. Авторы решили посмотреть, что это за участок РНК, который так сильно возрастает
	у фага при сильном свете. Они обнаружили, что это участок РНК, ответственный за белок psbA, который оказался общим у
	фага и его хозяина. Авторы предположили, что это вполне возможно, если фаги позаимствовали гены у хозяев в ходе
	рекомбинации.

\newpage
\section{Цель и задачи} \label{sec:code}

Эти статьи – хорошие примеры, но они не единственные в своём роде. Нам стали интересны такие события, и в нашей работе
мы предположили, что такие схожие вышестоящие регуляторные участки в ДНК фагов и их хостов могут встречаться не только у
цианобактерий и их фагов и T4 с E. Coli, а у всех фагов и их хостов.
В том числе, мы посчитали, что у многих фагов и их хозяев могут встречаться общие белки. Таким образом, если мы сможем
установить взаимоотношения фагов и их хозяинов, мы в будущем сможем прогнозировать отношения фагов и каких-либо хостов.

\newpage
\section{Обзор литературы} \label{sec:math}
\subsection{Hinton D.M.}
    \par{Контроль транскрипции имеет решающее значение для правильной экспрессии генов и упорядоченного развития. В течение
    многих лет бактериофаг Т4 обеспечивал простую модельную систему для исследования механизмов, регулирующих этот
    процесс. Развитие Т4 требует транскрипции ранних, средних и поздних РНК. Поскольку Т4 не кодирует свою собственную
    РНК-полимеразу, он должен перенаправить полимеразу своего хозяина, E.coli, в нужный класс генов в нужное время. Т4
    достигает этого за счет действия факторов, кодируемых фагами. Здесь я рассматриваю недавние исследования, изучающие
    транскрипцию пререпликативных генов T4, которые экспрессируются как ранние и средние транскрипты. Ранние РНК
    генерируются сразу после заражения из промоторов Т4, которые содержат отличные последовательности распознавания для
    полимеразы хозяина. Следовательно, ранние промоторы чрезвычайно хорошо конкурируют с промоторами-хозяевами за
    доступную полимеразу. Ранняя промоторная активность T4 дополнительно усиливается действием белка T4 Alt, компонента
    головки фага, который вводится в кишечную палочку вместе с ДНК фага. Alt модифицирует Arg265 на одной из двух
    субъединиц а РНК-полимеразы. Хотя работа с промоторами хозяина предсказывает, что эта модификация должна снизить
    активность промотора, транскрипция некоторых ранних промоторов Т4 увеличивается, когда РНК-полимераза модифицируется
    Алт. Транскрипция промежуточных генов Т4 начинается примерно через 1 минуту после заражения и протекает двумя путями:
    1) распространение ранних транскриптов в нижестоящие средние гены и 2) активация средних промоторов Т4 посредством
    процесса, называемого присвоением сигмы. При этой активации коактиватор Т4 AsiA связывается с регионом 4 \(\sigma^{70}\), sсубъединицей специфичности РНК-полимеразы. Это связывание резко реконструирует эту часть \(\sigma^{70}\), что затем позволяет активатору Т4 MotA также взаимодействовать с \(\sigma^{70}\). Кроме того, перестройка \(\sigma^{70}\) предотвращает формирование нормальных контактов области 4 с областью -35 промоторной ДНК, что, в свою очередь, позволяет MotA взаимодействовать со свом сайтом связывания ДНК, коробкой MotA, центрированной в области -30 средней промоторной ДНК. Присвоение сигмы T4 показывает, как специфический домен в РНК-полимеразе может быть переформован, а затем использован для изменения специфичности промотора. \cite{hinton}}
    \par{
    Фоновая экспрессия генома Т4-это строго регулируемый и элегантный процесс, который начинается сразу после заражения
    хозяина. Основной контроль над этой экспрессией происходит на уровне транскрипции. Т4 не кодирует свою собственную
    РНК-полимеразу (РНАП), а вместо этого кодирует множество факторов, которые служат для изменения специфичности
    полимеразы по мере развития инфекции. Эти изменения коррелируют с временной регуляцией трех классов транскрипции:
    ранней, средней и поздней. Ранняя и средняя РНК обнаруживается дорепликативно
    \cite{hinton1,hinton2,hinton3,hinton4,hinton5,hinton6}, в то время как поздняя транскрипция сопровождается
    репликацией Т4 и обсуждается в другой главе. Ранние транскрипты T4 генерируются из ранних промоторов (Pe),которые активны сразу после заражения. Ранняя РНК обнаруживается даже в присутствии хлорамфеникола, антибиотика, который
    предотвращает синтез белка. Напротив,средние транскрипты Т4 генерируются примерно через 1 минуту после заражения при
    \(37^\circ C\)  и требуют синтеза фагового белка. Средняя РНК синтезируется двумя способами: 1) активация средних
    промоторов (Pm) и 2) расширение транскриптов Pe из ранних генов в нижестоящие средние гены. Этот обзор посвящен
    исследованиям ранней и средней транскрипции T4, начиная с тех, которые подробно описаны в последней книге T4
    \cite{hinton1}-\cite{hinton5}. На момент этой публикации ранние и средние транскрипты были подробно охарактеризованы,
    но механизмы, лежащие в основе их синтеза, только зарождались. В частности, эксперименты in vitro только что
    показали, что для активации средних промоторов требуется модифицированный Т4 РНАП и активатор Т4 МотА
    \cite{hinton7,hinton8}. Последующая работа определила необходимую модификацию RNAP как плотное связывание белка 10
    кДа, AsiA, с субъединицей \(\sigma^{70}\) RNAP \cite{hinton9}-\cite{hinton13}. Кроме того, в настоящее время стало
    доступно множество структурной и биохимической информации о RNAP E. coli [рассмотрено в [14-16]], MotA и AsiA
    [рассмотрено в \cite{hinton2}]. Как подробно описано ниже, теперь у нас есть гораздо более механистическое понимание
    процесса дорепликативной транскрипции Т4. Чтобы понять этот процесс, мы сначала начнем с обзора транскрипционного
    механизма хозяина и RNAP.} \\ 
    {\Large Механизм транскрипции кишечной палочки }
    \par{Голофермент E. coli RNAP, как и все бактериальные RNAP, состоит из ядра субъединиц \((\beta,\beta',\alpha_1,\alpha_2\) и \(\omega)\), которое содержит активный сайт для синтеза РНК, и фактора специфичности \(\sigma\), который распознает промоторы в ДНК и устанавливает начальный сайт для транскрипции. Первичный \(\sigma\), \(\sigma^{70}\) в E. coli, используется во время экспоненциального роста; альтернативные факторы \(\sigma\) направляют транскрипцию генов, необходимых в различных условиях роста или во время стресса [рассмотрено в [17-19]]. Анализ последовательности/функций сотен факторов \(\sigma\) выявил различные регионы и субрегионы сохранения. Большинство \(\sigma\)-факторов имеют сходство в областях 2-4, центральной через С-концевую часть белка, в то время как первичные \(\sigma\)-факторы также имеют родственную N-концевую часть, область 1.}
    \par{Недавняя структурная информация, наряду с предыдущими и текущими биохимическими и генетическими работами
    [рассмотренными в[14,15,20,21]], привела к биомолекулярному пониманию функции RNAP и процесса транскрипции. В
    настоящее время доступны структуры голофермента, ядра и частей первичных \(\sigma\) термофильных бактерий с ДНК и без
    ДНК [15,16,22-28], а также структуры областей только E. coli \(\sigma^{70}\) [29] и в комплексе с другими белками
    [26,30]. Эта работа показывает, что взаимодействие между \(\sigma^{70}\) и ядром внутри голофермента RNAP является
    обширным (рис. 1). Она включает в себя контакт между частью области \(\sigma\) 2 и спиральным/спиральным
    доменом,состоящим из \(\beta\), \(\beta'\), взаимодействие области \(\sigma^{70}\) 1.1 в“ челюстях” в нисходящем
    канале ДНК (где ДНК ниже по течению от места начала транскрипции будет расположена, когда RNAP связывает промотор), а
    также взаимодействие между областью \(\sigma^{70}\) 4 и частью субъединицы \(\beta\), называемой \(\beta\)-лоскутом.}
    \par{Для начала транскрипции части RNAP должны сначала распознать и связать с элементами распознавания двухцепочечной
    (ds) ДНК, присутствующими в промоторной ДНК(рис. 1) [рассмотрено в [20]]. Каждый из C-концевых доменов субъединиц
    \(\alpha\) (\(\alpha\)-CTDS) может взаимодействовать с элементом UP, богатыми последовательностями A/T,
    присутствующими между позициями -40 и -60. Части \(\sigma^{70}\), присутствующие в RNAP, могут взаимодействовать с
    тремя различными элементами dsDNA. Спираль-поворот-спираль, мотив связывания ДНК в области \(\sigma^{70}\) 4 может
    связываться с элементом -35, область \(\sigma^{70}\) 3 может связываться с последовательностью -15 TGn -13 (TGn), а
    субрегион \(\sigma^{70}\) 2.4 может связываться с позициями -12/-11 элемента -10. Распознавание элемента -35 также
    требует контакта между остатками в области 4 \(\sigma^{70}\) и заслонкой \(\beta\), чтобы правильно расположить
    \(\sigma^{70}\) для одновременного контакта элемента -35 и нижестоящих элементов. Как правило, промотор должен
    содержать только два из трех \(\sigma^{70}\)-зависимых элементов для активности; таким образом, E. промоторы coli
    могут быть свободно классифицированы как -35/-10 (основной класс), TGn/-10 (также называемый расширенным -10) или
    -35/TGn[рассмотрено в [20]].}
    \par{Первоначальное связывание RNAP с элементами промотора dsDNA обычно приводит к нестабильному, “закрытому”
    комплексу(RPc) (рис. 1). Создание стабильного, “открытого” комплекса(RPo) требует изгиба и разматывания ДНК [31] и
    основных конформационных изменений (изомеризации) полимеразы (рис.1) [32,33]; рассмотрено в[20]]. В RPo размотка ДНК
    создает пузырь транскрипции от -11 до ~+3, обнажая одноцепочечный(ss) шаблон ДНК для транскрипции. Добавление
    рибонуклеозидтрифосфатов (RNTPS) затем приводит к синтезу РНК, которая остается в виде гибрида ДНК/РНК в течение
    примерно 8-9 лет. Генерация более длинной РНК инициирует экструзию РНК через выходной канал РНК, образованный частями
    \(\beta\) и \(\beta'\) внутри ядра. Поскольку этот канал включает в себя связанный с \(\sigma^{70}\)
    \(\beta\)-лоскут, считается, что прохождение РНК через канал помогает высвобождению \(\beta\) из ядра, облегчая
    клиренс промотора. Полученный комплекс элонгации, EC, содержит основную полимеразу, шаблон ДНК и синтезированную РНК
    (рис. 1) [рассмотрено в [34]]. ЭК быстро перемещается по ДНК со скоростью около 50 нтл/сек, хотя комплекс может
    останавливаться в зависимости от последовательности [35]. Прекращение транскрипции происходит либо при внутреннем
    сигнале прекращения, структуре стволовой петли (шпильки), за которой следует U-богатая последовательность, либо при
    Rho-зависимом сигнале прекращения [рассмотрено в [36,37]]. Образование шпильки РНК внутренней последовательностью
    терминатора может способствовать прекращению путем дестабилизации гибрида РНК/ДНК. Rho-зависимая терминация
    опосредуется взаимодействием белка Rho с участком колеи (последовательностью использования Rho), неструктурированной,
    иногда богатой C последовательностью, которая находится выше по течению от места терминации. После связывания с РНК
    Rho использует гидролиз АТФ для транслокации вдоль РНК, догоняя EC в месте паузы. Как именно Ро разъединяет
    приостановленный комплекс, до конца не понятно; ДНК:РНК-геликазная активность Rho может обеспечить силу для
    “выталкивания” RNAP из ДНК. Только Rho достаточно для завершения в некоторых зависимых от Rho местах завершения.
    Однако на других участках процесс прекращения также нуждается в вспомогательных белках E. coli NusA и/или NusG
    [рассмотрено в [36]].}
    \par{При наличии в межгенных регионах участки колеи легко доступны для взаимодействия с Rho. Однако, когда они
    присутствуют в областях, кодирующих белки, эти сайты могут быть замаскированы путем трансляции рибосом. В этом случае
    прекращение Rho не наблюдается, если только вышестоящий ген не переведен, например, когда мутация породила
    нонсенс-кодон. В таком случае Rho-зависимая терминация может предотвратить распространение транскрипции в нисходящий
    ген. Таким образом, в этой ситуации, которая называется полярностью [38], предотвращается экспрессия как восходящего
    мутированного гена, так и нисходящего гена.}

\newpage
\section{Материалы и Методы} \label{sec:code}
\subsection{Программные средства}
\begin{itemize}
    \item Virus-Host DB - База данных Virus-Host организует данные о взаимоотношениях между вирусами и их хостами,
    представленные в виде пар идентификаторов таксономии NCBI для вирусов и их хостов. База данных Virus-Host охватывает
    вирусы с полными геномами, хранящимися в 1) NCBI/RefSeq и 2) GenBank, номера присоединения которых указаны в геномах
    EBI. Информация о хосте собирается из RefSeq, GenBank (в свободном текстовом формате), UniProt, ViralZone и вручную
    курируется с дополнительной информацией, полученной в результате опросов литературы. \cite{virus-host}
    
    \item GeneMarkS-2 - программное обеспечение для поиска генов внутри геномов. \cite{lomsad}
    
    \item NCBI Entrez Programming Utilities - программное обеспечение, набор из восьми серверных программ, которые
    обеспечивают стабильный интерфейс с системой запросов и баз данных Entrez в Национальном центре биотехнологической
    информации (NCBI) \cite{entrez}
    
    \item Biopython - набор свободно доступных инструментов для биологических вычислений, написанных на Python
    международной командой разработчиков. \cite{biopython}
    
    \item The MEME Suite - программное обеспечение для поиска новых, нераскрытых мотивов (повторяющихся паттернов
    фиксированной длины) в последовательностях. МЕМ разбивает паттерны переменной длины на два или более отдельных
    мотива. \cite{bailey}
    
    \item Python 3.9.4 - среда для запуска скриптов формата *.py \cite{python}
    
    \item Jupyter Notebook -веб-приложение с открытым исходным кодом, которое позволяет создавать и обмениваться
    документами, содержащими живой код, уравнения, визуализации и повествовательный текст. Использование включает в себя:
    очистку и преобразование данных, численное моделирование, статистическое моделирование, визуализацию данных, машинное
    обучение и многое другое. \cite{jupyter}
    
    \item Скрипты отсюда \(https://github.com/poolsar42/phages-and-hosts\) - подумать, что про них написать!
    \cite{github}
    
    \item \(Fasta\_merge.ipynb\) - скрипт для объединения регуляторных областей хозяина и его фагов в один файл.
    \item \(Hosts\_upstream\_regions.ipynb\) - скрипт для выделения регуляторных областей.
    \item \(Proteins\_formation.ipynb\) - скрипт для сбора белковых последовтельностей в один файл.
    \item \(Searching\_host\_genomes.ipynb\) - скрипт для выгрузки необходимых к работе геномов хостов.
    \item \(Unpacking.ipynb\) - скрипт для распаковки выгруженных геномов.
    \item \(VH\_tsv\_writer.ipynb\) - скрипт, выписывающий в файл ID хоста и ID прилежащих к нему фагов.
    \item \(Virus\_genes.ipynb\) - скрипт для выгрузки геномов фагов.
    \item \(missing\_phage\_genomes.ipynb\) - скрипт для поиска фагов, для которых GMS2 не находит гены.
    \item \(proteins\_convertation.ipynb \) - скрипт для преобразования \\ ДНК-последовательностей
                                в белковые.
    \item \(regex\_promoter.ipynb \) - скрипт для анализа полученных результатов
    \item \(run\_gms2.py\) - скрипт для поиска генов в геномах
    \item \(tsv\_write.ipynb\) - скрипт для выписывания информации о связи между хостами и фагами в один файл
\end{itemize}

\subsection{Получение общих мотивов}
    \subsubsection{Протокол выгрузки геномов}
    \begin{itemize}
        \item С FTP сервера Virus-Hist DataBase выгрузить файл virushostdb.tsv - в нём содержится информация о связи между фагами и их хостами.
    
        \item Запустить скрипт \(tsv\_write.ipynb\) - выписывает в удобном формате все взаимоотношения между фагами и хостами.
    
      \item Запустить скрипт \(Virus\_genes.ipynb\) - для выгрузки геномов фагов.
        \item Запустить скрипт \(Searching\_host\_genomes.ipynb\) - для выгразки геномов хостов.
        \item Запустить скрипт \(Unpacking.ipynb\).

    \end{itemize}

    \subsubsection{Протокол получение вышестоящих регуляторных регионов}
        \begin{itemize}
            \item Запустить скрипт \(run\_gms2.py\)
            \item Запустить скрипт \(Hosts\_upstream\_regions.ipynb\)
        \end{itemize}
    \subsubsection{Протокол подготовки данных к поиску мотивов}
        \begin{itemize}
            \item Запустить скрипт \(Fasta\_merge.ipynb\)
            \item Запустить MEME Suite с получившимися файлами.
        \end{itemize}

\par{Нам нужно было посмотреть на сходтсва между вышестоящими регуляторными участками у хостов и фагов. Для этого нужно
было посмотреть на геномы хостов и фагов. Мы собрали геномы всех фагов, которые находятся в базе данных Virus-Host DB,
там для каждого фага также был указан или были указаны хосты, к которым он может прикрепиться. Их RefSeq ID. Мы всё
выгрузили с FTP сервера.}

\par{Далее мы выгрузили все геномы хостов с базы данных RefSeq NCBI. Когда у нас были геномы фагов и их хостов, нам было
нужно получить с этих геномов гены. Гены мы находили с помощью Gene-Marks 2. Что хорошо, Gene-Marks 2 также дал позицию
каждого гена в геноме. Каждому гену посчитали его транслирующийся белок, также с помощью Gene-Marks 2. После этого для
каждого гена нашли вышестоящие регуляторные области, длиной в 50 нуклеотидов, которые нужны для посадки РНК-полимераз(?).
Наша гипотеза как раз заключается в том, что эти вышестоящие области у фага и его хозяина для некоторых генов будут
совпадать по некоторым мотивам.}

\par{Мы поработали с последовательностями. Мы выгружали геномы всех фагов и хостов отдельными файлами. Мы предварительно
обработали файлы, достали из них вышестоящие регуляторные области длинной в 50 нуклеотов. Вышестоящие области хоста и
всех фагов, прилегающих к нему, мы закинули в один файл - в этом файле мы и искали общие мотивы.}

\par{Далее проверку на совпадения вышестоящих областей мы засунули все вышестоящие области хоста и его фагов в программу
MEME Suite. Как и ожидалось, мы получили множество профилей, которые были общими у фагов и их хоста. }

\par{Также MEME-Suite для каждого профиля мотива дала нам PFM (Position Frequency Matrix), поэтому у нас была возможность
сравнить сами мотивы между собой. У нас появилась гипотеза, что мотивы фагов, атакующих одного хоста, могут быть также
схожи и как-то коррелировать между собой. Мы решили посмотреть корреляцию 2D массивов - потому что для каждой позиции в
матрице частот у нас было 4 значения - на 4 возможных нуклеотида.}

\newpage
\section{Результаты и Обсуждение}

\newpage
\section{Выводы}


\newpage
\begin{thebibliography}{9}
    \addcontentsline{toc}{section}{\refname}
    \bibitem{hinton1} Stitt B, Hinton DM: Regulation of middle-mode transcription.In Molecular biology  of  bacteriophage  T4.Edited by: Karam JD, Drake J, Kreuzer KN, Mosig G, Hall D, Eiserling F, Black L, Spicer E, Kutter E, Carlson K, Miller ES.Washington, D.C.: American Society for Microbiology; 1994:142-160
    \bibitem{hinton2} Brody E, Rabussay D, Hall D:Regulation of transcription of prereplicative genes. In Bacteriophage  T4.Edited by: Mathews CK, Kutter EM, Mosig G,Berget PB. Washington, D. C.: American Society for Microbiology;1983:174-183
    \bibitem{hinton3} Hinton DM, Pande S, Wais N, Johnson XB, Vuthoori M, Makela A, Hook-Barnard I: Transcriptional takeover by sigma appropriation: remodelling of the sigma70 subunit of Escherichia coli RNA polymerase by the bacteriophage T4 activator MotA and co-activator AsiA.Microbiology2005,151:1729-1740.
    \bibitem{hinton4} Miller ES, Kutter E, Mosig G, Arisaka F, Kunisawa T, Ruger W: BacteriophageT4 genome. Microbiol  Mol  Biol  Rev2003,67:86-156
    \bibitem{hinton5} Wilkens K, Ruger W: Transcription from early promoters. In Molecular Biology  of  Bacteriophage  T4.Edited by: Karam JD, Drake JW, Kreuzer KN,Mosig G, Hall DH, Eiserling FA, Black LW, Spicer EK, Kutter E, Carlson K,Miller ES. Washington, D. C.: American Society for Microbiology;1994:132-141.
    \bibitem{hinton6} Weisberg R, Hinton DM, Adhya S: Transcriptional Regulation in Bacteriophage. In Encyclopedia  of  Virology.3 edition. Edited by: Mahy BWJ,van Regenmortel MHV. Oxford: Elsevier; 2008:174-186, 174-186.
    \bibitem{hinton7} Hinton DM: Transcription from a bacteriophage T4 middle promoter using T4 motA protein and phage-modified RNA polymerase.J  BiolChem 1991, 266:18034-18044
    \bibitem{hinton8} Schmidt RP, Kreuzer KN: Purified MotA protein binds the -30 region of a bacteriophage T4 middle-mode promoter and activates transcription invitro. J  Biol  Chem1992, 267:11399-11407
    \bibitem{hinton9} Stevens A: New small polypeptides associated with DNA-dependent RNA polymerase of Escherichia coli after infection with bacteriophage T4. Proc  Natl  Acad  Sci  USA1972, 69:603-607.
    \bibitem{hinton10} Stevens A:An inhibitor of host sigma-stimulated core enzyme activitythat purifies with DNA-dependent RNA polymerase of E. coli followingT4 phage infection.Biochem  Biophys  Res  Commun1973,54:488-493.
    \bibitem{hinton11} Ouhammouch M, Orsini G, Brody EN: The asiA gene product of bacteriophage T4 is required for middle mode RNA synthesis.J  Bacteriol1994, 176:3956-3965
    \bibitem{hinton12} Ouhammouch M, Adelman K, Harvey SR, Orsini G, Brody EN: BacteriophageT4 MotA and AsiA proteins suffice to direct Escherichia coli RNA polymerase to initiate transcription at T4 middle promoters. Proc  NatlAcad  Sci  USA1995, 92:1451-1455.
    \bibitem{hinton13} Hinton DM, March-Amegadzie R, Gerber JS, Sharma M: Bacteriophage T4middle transcription system: T4-modified RNA polymerase; AsiA, a sigma70 binding protein; and transcriptional activator MotA.Methods  Enzymol1996, 274:43-57
    \bibitem{virus-host} https://www.genome.jp/virushostdb/ 
    \bibitem{biopython} https://biopython.org/
    \bibitem{python} https://docs.python.org/3/
    \bibitem{jupyter} https://jupyter.org/
    \bibitem{github} https://github.com/poolsar42/phages-and-hosts
    \bibitem{entrez} https://www.ncbi.nlm.nih.gov/books/NBK25501/
    \bibitem{hinton} Hinton D.M. Transcriptional control in the prereplicative phase of T4 development // Virology
    journal. 2010. V. 7. P. 289. https://doi.org/10.1186/1743-422X-7-289
    \bibitem{puxty-evanx} Puxty, R.J., Evans, D.J., Millard, A.D. et al. Energy limitation of cyanophage development:
    implications for marine carbon cycling // ISME J. 2018, 12, 1273–1286. https://doi.org/10.1038/s41396-017-0043-3
    \bibitem{lomsad} Lomsadze, A., Gemayel, K., Tang, S., and Borodovsky, M. Modeling leaderless transcription and
    atypical genes results in more accurate gene prediction in prokaryotes // Genome research,  2018. 28(7), 1079–1089.
    https://doi.org/10.1101/gr.230615.117
    \bibitem{bailey} Bailey, T.L., Johnson, J., Grant, C.E., and Noble, W.S. The MEME Suite // Nucleic acids research,
    2015, 43(W1), W39–W49. https://doi.org/10.1093/nar/gkv416
    \bibitem{ryasik} Ryasik, A., Orlov, M., Zykova, E., Ermak, T., Sorokin, A. Bacterial promoter prediction: Selection
    of dynamic and static physical properties of DNA for reliable sequence classification // 2018
    https://www.worldscientific.com/doi/abs/10.1142/S0219720018400036
    
\end{thebibliography}

\end{document}

\usepackage[english,russian]{babel}