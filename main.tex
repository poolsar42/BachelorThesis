%\documentclass[a4paper,12pt]{article}
\documentclass[14pt]{extarticle}

\usepackage{cmap}		
\usepackage[utf8]{inputenc}			
\usepackage[english,russian]{babel}
\usepackage{framed}
\usepackage{hyperref}
\usepackage{amsmath}
\usepackage{graphicx}
\usepackage[colorinlistoftodos]{todonotes}
\usepackage{wrapfig}
\usepackage{lipsum}
\usepackage{listings}
\usepackage{color}
\usepackage{indentfirst}
%\usepackage{times}
\usepackage{textcomp}
\usepackage{indentfirst}
\textheight=24cm % высота текста
\textwidth=16,5cm % ширина текста
%\oddsidemargin=30 mm %отступ от левого края
\oddsidemargin=0pt % отступ от левого края
\parindent=1,25cm
\topmargin=-2cm % отступ от верхнего края
\definecolor{mygray}{rgb}{0.4,0.4,0.4}
\definecolor{mygreen}{rgb}{0,0.8,0.6}
\definecolor{myorange}{rgb}{1.0,0.4,0}

\lstdefinestyle{customc}{
  belowcaptionskip=1\baselineskip,
  breaklines=true,
  frame=L,
  xleftmargin=\parindent,
  language=C,
  showstringspaces=false,
  basicstyle=\footnotesize\ttfamily,
  keywordstyle=\bfseries\color{green!40!black},
  commentstyle=\itshape\color{purple!40!black},
  identifierstyle=\color{blue},
  stringstyle=\color{orange},
  numbers=left,
  numbersep=13pt,
  numberstyle=\small\color{mygray},
}
\lstset{escapechar=@,style=customc}
\linespread{1.5}
\newcommand{\HRule}{\rule{\linewidth}{0.5mm}}

\begin{document}

%\begin{titlepage}
\begin{center}

\textsc{\Large Московский Физико-Технический Институт}\\
\textsc{\large (Национальный Исследовательский Университет)}\\[1.5cm]

% Upper part of the page. The '~' is needed because \\
% only works if a paragraph has started.
\includegraphics[width=0.25\textwidth]{img/logo.png}~\\[1cm]

\textsc{\Large Бакалаврская работа}\\[0.5cm]

% Title
\HRule \\[0.4cm]
{ \LARGE \bfseries Влияние вышестоящих регуляторных регионов и промоторов на связь между бактериофагами и их хозяевами \\[0.4cm] }

\HRule \\[1.5cm]

% Author and supervisor
\noindent
\begin{minipage}{0.4\textwidth}
\begin{flushleft} \large
\emph{Студент:}\\
Шарафутдинов Эмиль
\end{flushleft}
\end{minipage}%
\begin{minipage}{0.4\textwidth}
\begin{flushright} \large
\emph{Преподаватель:} \\
 Родригес Валера Франсиско Эдуардо
\end{flushright}
\end{minipage}

\vfill

% Bottom of the page
{\large 24 апреля 2021 г.}

\end{center}
\end{titlepage}


\newpage
\tableofcontents

\newpage
\begin{center}
\item \section{Аннотация}
\end{center}

\par{Бактериофаги (фаги) - это вирусы, которые избирательно поражают бактериальные клетки. Такая специфика обусловлена
зависимостью фага от клеточного механизма хозяина, который нужен фагу для репликации и биосинтеза вирусных компонентов.
Пример такой зависимости  можно привести для фага Т4. Во время транскрипции, ранняя конкуренция за РНК-полимеразу
хозяина позволяет вирусу полностью захватить механизм репликации путем сигма-присвоения(sigma appropriation) [1].
Недавно показали[2], что фаги, которые заражают цианобактерии, кодируют ген фотосинтеза. Причём транскрипция этого гена
реагирует на те же условия окружающей среды, что и ген, который кодируется у хозяина. Из этого можно предположить, что
существует связь между регуляторными элементами фага и регуляторными элементами их хозяина. Это поможет лучше
предсказывать взаимоотношения фаг-хозяин.}



\newpage
\begin{center}
\item \section{Введение}
\end{center}

    \begin{center}
    \item \subsection{Бактериофаги}
    \end{center}
    
    \begin{figure}[h]
        \centering
            \includegraphics[width=\textwidth]{img/average_phage.jpg}
        \caption{Как фаг убивает бактерию. (1) Фаг сначала приземляется на бактерии. (2) Затем он вводит свою ДНК внутрь
        бактерий. (3) ДНК копируется и используется для изготовления капсидов для новых фагов. (4) Новые фаги собираются
        и взрывают бактерию, убивая её в процессе. \cite{advdisphage}}
        \label{fig:skybox}
    \end{figure}
    
	\par{Бактериофаг - это вирус, который заражает и размножается внутри бактерии и археи. Как правило, бактериофаг 
	состоит из белковой оболочки и генетического материала одноцепочечной или двуцепочечной нуклеиновой кислоты (ДНК 
	или, реже, РНК). Фаги размножаются внутри бактерии после введения своего генома в ее цитоплазму. 
	\cite{phagewikieng,phagewikiru}}
	
	\par{Более 90\% бактериофагов имеют большие геномы в виде двухцепочечных ДНК. Они относятся к трем основным 
	морфологическим группам бактериофагов. Это Myoviridae (с длинными, жесткими, сократительными хвостами), Siphoviridae
	(с длинными, гибкими, неконтрактирующими хвостами) и Podoviridae (с короткими, неконтрактирующими хвостами). 
	\cite{phageapps}}
	
	\par{Для бактерии исход заражения бактериофагом может быть разным. Некоторые бактериофаги вызывают лизис и гибель 
	клетки в течение очень короткого времени. Как правило, это приводит к высвобождению сотен новы вирусов в течение 
	нескольких минут или часов. Это процесс, который может повторяться до тех пор, пока их бактериальный хозяин 
	присутствует в достаточном количестве для поддержки репликации. Многие бактериофаги ведут себя именно так (иногда их
	называют вирулентными) и не способны продуцировать какой-либо другой вид инфекции. Есть также умеренные 
	бактериофаги, которые заражают клетки, а затем становятся спящими в латентном состоянии; реплицируются вместе с 
	хромосомой хозяина и впоследствии передаются каждой дочерней клетке после деления. Однако эти дремлющие фаги могут 
	быть активированы несколько причинами, например повреждением ДНК. Для некоторых бактериофагов хромосомная ДНК 
	хозяина может быть упакована в частицы бактериофага во время репликации бактериофага вместо генома бактериофага. Это
	может привести к высокому уровню горизонтального переноса генов в бактериальной популяции.  \cite{phagetreat}}
    
    \begin{center}
    \item \subsubsection{Использование бактериофагов}
    \end{center}
    
    \par{Использование бактериофагов в качестве высокоспецифичных антимикробных средств широко документировано в 
    литературе. Их даже можно использовать в качестве регулируемого терапевтического средства. \cite{phagemed}}
    
    \par{Как уже было сказано, в большинстве случаев, бактериофаги не дают бактерии размножаться, а вместо этого она 
    производит дополнительные фаги \cite{phagetherapy}. В то время, когда по каким-либо причинам химические антибиотеки 
    могут не сработать, фаги могут быть хорошей заменой для них \cite{phageapps}. Преимущества включают снижение 
    побочных эффектов и снижение риска развития резистентности бактерии. Недостатки включают трудность поиска 
    эффективного фага для конкретной инфекции. \cite{advdisphage}}
    
    \par{Фаги применяются не только в медицине. Они могут использоваться в пищевой промышленности \cite{phagesalm}, их 
    используют в качестве противодействия биологическому оружию и токсинам, в диагностике (например, для поиска 
    стафилококка в крови), в молочной промышленности. У фагов также широко используются в разных биологических 
    исследованиях \cite{phagewikieng}}
    
    \begin{center}
    \item \subsubsection{Мотивы}
    \end{center}
    
    \par{Мотив последовательности ДНК, РНК или белка - это короткий паттерн, который сохраняется во время эволюции. 
    Мотивом в ДНК может может быть сайт связывания белка; в белках мотив может быть активным сайтом фермента или 
    структурной единицей, необходимой для правильного сворачивания белка. Мотивы в последовательностях являются одними 
    из основных функциональных единиц молекулярной эволюции.}
    
    \par{Многие геномы фагов кодируют небольшие белки, которые специфически изменяют РНК-полимеразу (RNAP)
    бактериального хозяина, ингибируя транскрипцию бактериальной ДНК и способствуя регулируемой транскрипции фаговой 
    ДНК. \cite{phagetrans}}    
    
    \par{Транскрипция бактерии начинается со связывания $\sigma$-фактора с каталитическим ``ядром'' RNAP. Начальный 
    $\sigma$-фактор, $\sigma^{70}$, специфично связывается с отдельными нуклеотидными последовательностями(мотивами) в 
    промоторе. Я описываю этот процесс для E.coli детальнее ниже. Фаги развили механизмы, которые модифицируют 
    $\sigma^{70}$ и перенаправляют бактериальную RNAP для транскрипции генов фага.}
    
\newpage
\begin{center}
\item \section{Цель и задачи} \label{sec:code}
\end{center}

    \par{Как было показано, фаги имеют широкое применение в самых разных областях. Понимание устройства фагов в природе и принципов их работы может упростить и/или улучшить способы их использования. В этой работе мы хотим разобраться, каким образом устроено взаимодействие между большинством фагов и их хозяевами. Нам интересно, наличие общих мотивов между фагами и их хозяевами в промоторных участках последовательностей ДНК. Также нам интересно, есть ли общие гены, экспрессирующиеся у этих фагов. В том числе, нам интересно проверить, есть ли общие мотивы в промотрных последовательностях между фагами.}

\newpage
\begin{center}
\item \section{Обзор литературы} \label{sec:math}
\item \subsection{Взаимодействие фага Т4 и E.coli}
\end{center}
        
        \par{Экспрессия генома Т4 - это строго регулируемый процесс, который начинается сразу после заражения хозяина. 
        Основной контроль над этой экспрессией происходит на уровне транскрипции. Т4 не кодирует свою собственную 
        РНК-полимеразу (RNAP), а вместо этого кодирует множество факторов, которые служат для изменения специфичности 
        полимеразы хозяина по мере развития инфекции. Изменения полимеразы хозяина сопровождаются с тремя классами 
        транскрипции фага: ранней, промежуточной и поздней. Ранняя и промежуточная РНК (РНК, образовавшаяся при ранней и
        промежуточной транскрипции) обнаруживается дорепликативно 
        \cite{hinton1,hinton2,hinton3,hinton4,hinton5,hinton6}, в то время как поздняя транскрипция сопровождается 
        репликацией Т4 и нам в нашей работе не интересна. Ранние транскрипты T4 генерируются из ранних промоторов 
        (Pe),которые активны сразу после инфицирования. Промежуточные транскрипты Т4 генерируются примерно через 1 
        минуту после заражения при \(37^\circ C\) и требуют синтеза белка фага. Средняя РНК синтезируется двумя 
        способами: 1) активация промежуточных промоторов (Pm) и 2) расширение транскриптов Pe из ранних генов в 
        нижестоящие средние гены.}
        
        \begin{center}
        \item \subsubsection {Начало транскрипции E.coli}
        \end{center}
        
        \par{E. coli RNAP холофермент, как и все бактериальные RNAP, состоит из ядра субъединиц 
        \((\beta,\beta',\alpha_1,\alpha_2\) и \(\omega)\), которое содержит активный сайт для синтеза РНК, и фактора 
        специфичности \(\sigma\), который распознает промоторы в ДНК и устанавливается на начальный сайт для 
        транскрипции. Первичный \(\sigma\), \(\sigma^{70}\) в E. coli, используется во время экспоненциального роста; 
        альтернативные факторы \(\sigma\) направляют транскрипцию генов, необходимых в различных условиях роста или во 
        время стресса \cite{17,18,19}}
        
        \par{Для начала транскрипции части RNAP должны сначала распознать и связаться с двухцепочечными (ds) элементами 
        распознавания ДНК, присутствующими в промоторе \cite{20}. Каждый из C-концевых доменов 
        $\alpha$-субъединиц ($\alpha$-CTDS) может взаимодействовать с вышестоящим элементом, A/T богатыми 
        последовательностями, присутствующими между позициями -40 и -60. Части \(\sigma^{70}\), присутствующие в RNAP, 
        могут взаимодействовать с тремя различными элементами dsDNA: с элементом -35, с последовательностью -15TGn-13 
        (TGn) и с позициями -12/-11 элемента -10. Как правило, промотор должен содержать только два из трех 
        \(\sigma^{70}\)-зависимых элементов для активности; таким образом, промоторы E. coli могут быть свободно 
        классифицированы как -35/-10 (основной класс), TGn/-10 (также называемый расширенным -10) или -35/TGn 
        \cite{20}.}
        
        \begin{center}
            \item \subsubsection{Этапы транскрипции фага Т4}
        \end{center}
        
        \begin{figure}[h]
            \centering
                \includegraphics[width=\textwidth]{img/Hinton.jpg}
            \caption{\textit{Сравнение последовательностей хоста E. coli, раннего Т4 и промежуточного промоторов Т4.}
            Сверху показаны последовательности и положения элементов распознавания промотора хозяина для
            $\sigma^{70}$-RNAP (UP, -35, TGn, -10)\cite{20,150}. Ниже аналогичные последовательности, обнаруженные в ранних
            \cite{4} и промежуточных \cite{91} промоторах T4, сходства выделены черным цветом, а различия-красным. W = A или T; R
            = A или G; Y = C или T, n = любой нуклеотид; заглавная буква представляет собой более высоко консервативное
            основание. \cite{hinton}}
            \label{fig:skybox}
        \end{figure}

        \par{Т4 заражает E.coli только во время экспоненциального роста. Транскрипция ранних генов Т4 начинается сразу 
        после заражения. Таким образом, для эффективной инфекции фаг должен быстро перенаправить 
        \(\sigma^{70}\)-ассоциированный RNAP, который активно участвует в транскрипции генома хозяина, на ранние 
        промоторы T4. Такое немедленное перенаправление часто происходит успешно отчасти потому, что большинство ранних 
        промоторов T4 содержат такие же элементы распознавания \(\sigma^{70}\)-RNAP (-35, TGn и -10 элементов) и 
        элементы \(\alpha\)-CTD UP (списки известных последовательностей ранних промоторов T4 в \cite{4, 5}). Тем не
        менее, выравнивание последовательностей ранних промоторов Т4 показало дополнительные области совпадения, и есть 
        предположение, что они содержат и другие последовательности, которые могут оптимизировать взаимодействие RNAP 
        хозяина с элементами промотора T4. Следовательно, в отличие от большинства промоторов хозяина, которые относятся
        к классам -35/-10, TGn/-10 или -35/TGn, ранние промоторы T4 могут быть описаны как ``супер'' UP/-35/TGn/-10 
        промоторы \cite{hinton}. Действительно, большинство ранних промоторов Т4 очень хорошо конкурируют с промоторами 
        хозяина за доступный RNAP \cite{39}.}
        
\begin{center}
    \item \subsection{Взаимодействие цианофага и цианобактерии}
\end{center}
    
    \par{В этой статье авторы инетересовались изменением климата. Изучение этого вопроса требуте детального знания о
    биологической трансформации углерода на Земле. Стало очевидно, что океан - это важный поглотитель
    атмосферного углекислого газа. В плане поглощения $CO_2$ в открытом океане доминируют доминируют два организма: 
    Prochlorococcus и Synechococcus \cite{2_,3_,4_}. Эти организмы стали моделью для изучения потока углерода из $CO_2$ 
    в микробную жизнь \cite{5_,6_}. Однако биологическик потери (антагонистические взаимодействия, выпас скота, вирусный лизис) прохлорококка и синехокка мало изученны. У вирусов только открывают новые гены, которые действовуют для поддержания фотосинтеза во время инфекции \cite{7_,8_}, так что, несмотря на окончательную потерю фиксированного углерода в растворенном органическом веществе в результате лизиса, фиксация $CO_2$ может поддерживаться временно в течение относительно длительных латентных периодов вируса. Недавно было показано, что на самом деле цианофаг отключает $CO_2$ инфекцию на ранних стадиях заражения, но при этом поддерживает реакции фотосинтеза \cite{9_}.}
    
    \begin{center}
        \item \subsubsection{Влияние интенсивности света на транскрипцию фага}
    \end{center}


    \begin{figure}[]
            \centering
            \includegraphics[width=\textwidth]{img/protein.jpg}
            \caption{\textit{Влияние интенсивности света на глобальную экспрессию генов.} \textbf{a} Вулкано-плот с
            логарифмической шкалой, показывающий относительное изменение транскриптом (относительно воздействия слабого
            света). Ось y показывает скорректированное p-значение ложного обнаружения, рассчитанное с помощью edgeR.
            Оранжевый круг показывает статистически значимую дифференциальную экспрессию генов по edgeR. \textbf{b}
            Гистограмма длин вышестоящих межгенных последовательностей в геноме цианофага S-PM2d. Ячейка, содержащая
            psbA, показана оранжевым цветом. \textbf{c} Мотивы вышестоящих последовательностей ДНК, обнаруженных у
            цианофаговых PSBA. Красные полосы указывают на -35 и -10 элементы сайтов связывания $\sigma^{70}$, а синяя
            полоса показывает сайт связывания для $\sigma$ Gp55. \textbf{d} Предсказываемая структура складывания петли
            шпильки S-PM2d psbA 5'-UTR \cite{puxty-evanx}}
            \label{fig:skybox}
    \end{figure}
    
    \par{Авторы обнаружили что скорость транскрипции фагов при сильном свете увеличивается, в то время, как репликации
    ДНК была такой же, как и при слабом. Авторы обнаружили, что скорость экспрессии только одного гена цианофага была
    пропорциональна интенсивности света. Это был фотосинтетический AMG psbA, кодирующий полипептид D1, расположенный в
    ядре центра реакции PSII. Synechococcus также кодируют это белок, но у них не было никаких увеличений в скорости 
    транскрипции. \cite{puxty-evanx}}
    
    \par{Цианобактерии, включая морской Synechococcus, демонстрируют светозависимую транскрипцию psbAs \cite{18_,23_}. В
    модельной цианобактерии на светозависимое изменение транскриптов psbA влияют несколько факторов, включая 
    альтернативные $\sigma$-факторы \cite{24_,25_}. Более того, продукты деградации D1 непосредственно связываются с 
    вышележащими регионами последовательностей psbA, так что вызванное светом повреждение может положительно влиять на 
    транскрипцию psbA. Авторы не смогли найти консервативные мотивы в вышележащих областях в последовательностях psbA 
    S-PM2d, которые были бы общими с Synechococcus.} 
    
    \par{Межгенную вышестоящая регуляторная последовательность у psbA нехарактерно длинная у S-PM2d(232 пар 
    нуклеотидов,по сравнению с медианой 6 по всему геному). Это также справедливо и для других цианофагов, где 
    вышестоящие регуляторные области psbA варьируют в пределах 125-453 пн. Также с высокой степенью достоверности  было 
    обнаружено между рассмотренными фагами 6 общих мотивов в этих последовательностях \cite{puxty-evanx}. Из них два 
    присутствовали в S-PM2d. Эти два мотива - это  -35 (мотив 6) и -10 (мотив 3) (на рисунке) элементы сайтов связывания
    фактора транскрипции $\sigma^{70}$, типичные для ранних генов Т4-подобных фагов \cite{31_}. К тому же, было 
    обнаружено, что как и у T4, мотив 3 содержит сайт для посадки позднего $\sigma$-фактора Gp55. Таким образом, у 
    S-PM2d имеются мотивы, необходимые для скоординированной экспрессии psbA на начальной и поздней стадиях инфекции.}
    
    \begin{center}
        \item \subsection{GeneMarkS-2}
    \end{center}
    
   
    
    \par{Для поиска прокариотических генов существует несколько инструментов (например GeneMarkS, Glimmer3, Prodigal), 
    которые известны достаточно высокой точностью в предсказании белок-кодирующих ORF. В среднем эти инструменты 
    способны найти более 97\% генов в проверенном тестовом наборе с точки зрения правильного предсказания концов гена 3'
    (Besemer et al. 2001; Delcher et al. 2007; Hyatt et al. 2010). Кроме того, точность определения стартов генов 
    составляет в среднем 90\% (Hyatt et al., 2010). Но основная часть генов, которые не были обнаружены, принадлежали в 
    основном к атипичной категории, т. е. гены с паттернами, не соответствующими видоспецифической натренированной 
    модели, обученной на основной части генома (Бородовский и др., 1995). Тем не менее авторы описали метод, с помощью 
    которого можно получить более высокую точность в обнаружении генов. \cite{lomsad}}
    
    \begin{figure}[]
            \centering
            \includegraphics[width=\textwidth]{img/gms2_1.jpg}
            \caption{Диаграмма основного состояния обобщенной скрытой марковской модели (GHHM) геномной 
            последовательности прокариот. Состояния, показанные на верхней панели, использовались для моделирования гена
            в прямом направлении. Гены в обратном направлении моделировались идентичным набором состояний (с обратными 
            направлениями перехода). Различные состояния, моделирующие гены в прямых и обратных цепочках, были связаны 
            через состояние межгенной области или же перекрывающимися участками в противоположных цепочках. 
            \cite{lomsad}}
            \label{fig:skybox}
    \end{figure}
    
    \par{GeneMarkS-2 использует довольно сложную модель гена (рис. 3). Большинство белок-кодирующих областей в 
    прокариотических геномах имеют видоспецифичные паттерны из нескольких олигонуклеотидов (например, кодонов) (Fickett 
    and Tung 1992). GeneMarkS-2 изучает эти паттерны и оценивает параметры типичной модели белок-кодирующих областей, 
    трехпериодической цепи Маркова (Бородовский и др.,1986a, b), итеративно самообученаясь на всем геноме.}
    
    
    \par{Алгоритм обучения без учителя выполняет несколько двухэтапных итераций (рис. 4). Каждая итерация приводит к (1)
    сегментации генома на кодирующие белки (CDS) и некодирующие области (прогнозирование генов) и (2) переоценке 
    параметров модели}
    
    \begin{figure}[]
            \centering
            \includegraphics[width=\textwidth]{img/gms2_2.jpg}
            \caption{Диаграма обучения без учителя для GMS-2 \cite{lomsad}}
            \label{fig:skybox}
    \end{figure}
    
    
    \par{\textit{На первой итерации} Алгоритм Витерби вычисляет максимально вероятную 
    последовательность скрытых состояний (рис. 1) вдоль генома. После первого запуска
    алгоритма Витерби все участки генома, помеченные как ``белок-кодирующие'', собираются в обучающий набор для оценки 
    параметров ``типичной'' (для данного генома) модели. Аналогично помеченные как ``некодирующие'' участки нужны для 
    оценки параметров модели некодирования, которая строется в виде однородной цепи Маркова второго порядка.}
    
    
    \par{\textit{На второй итерации} Модель уже выбирает последовательности, расположенные вокруг запусков генов, 
    и выводит модели паттернов, кодирующих регуляцию транскрипции и/или трансляции. \textit{Третью итерацию и 
    далее} GeneMarkS-2 продолжает шаги предсказаний/оценок до тех пор, пока не 
    будет выполнено условие сходимости (99\% идентичности в гене начинается между последовательными итерациями).}
    
    
    \par{Для нас было важно, что наблюдаемая частота ошибок GeneMarkS-2 составила 4,4\%, за ней последовали Prodigal на 
    уровне 6,1\%, GeneMarkS на уровне 10,2\% и, наконец, Glimmer3 на уровне 13,2\%. Таким образом, GeneMarkS-2 сделал 
    наибольшее количество правильных прогнозов среди четырех искателей генов. \cite{lomsad}}
    
    \begin{center}
        \item \subsection{The MEME-Suite}
    \end{center}
    
    \par{The MEME-Suite - это программируемый набор инструментов, которые анализируют мотивы последовательностей. Ядром 
    набора является алгоритм обнаружения мотивов \texttt{MEME}, который находит мотивы в невыравненных 
    последовательностях ДНК, РНК или белков \cite{_1}. Поисковик мотивов ищет \textit{новые} мотивы в предоставленных 
    последовательностях. Существует много инструментов, предназначенных для поиска мотивов только в ДНК; The MEME-Suite 
    может искать мотивы и сканирование последовательностей ДНК, РНК и белков. \cite{bailey}}
    
    \begin{figure}[]
        \centering
        \includegraphics[width=\textwidth]{img/meme.jpg}
        \caption{Что умеет делать The MEME-Suite. Нам нужно находить новые мотивы в последовательностях - это 
        указано сверху посередине. \cite{bailey}}
        \label{fig:skybox}
    \end{figure}
    
    \par{Как было описано выше, алгоритм \texttt{MEME}, обнаруживает один или несколько мотивов в наборе 
    последовательностей ДНК, РНК или белков, используя метод максимизации математического ожидания для подгонки 
    двухкомпонентной конечной модели к набору последовательностей. Путём подгонки модели к данным обнаруживается 
    единичный мотив, дальше модель стирает вхождения максимально вероятного мотива, найденного таким образом и процесс 
    повторяется. Таким образом ищется множество мотивов. У алгоритм два необходимых параметра - минимальная и 
    максимальная ширина искомых мотивов. Он возвращает профиль каждого мотива и порог, которые вместе могут быть 
    использованы в качестве байесовского оптимального классификатора для поиска вхождений мотива в других базах данных. 
    Алгоритм оценивает, сколько раз каждый мотив встречается в каждой последовательности в наборе данных, и выводит 
    частоту встречаемости этого мотива. \cite{meme}}
    
    
\newpage
\begin{center}
\item \section{Материалы и Методы} \label{sec:code}
\item \subsection{Программные средства}
\end{center}
\begin{itemize}
    \item Virus-Host DB - База данных Virus-Host организует данные о взаимоотношениях между вирусами и их хостами,
    представленные в виде пар идентификаторов таксономии NCBI для вирусов и их хостов. База данных Virus-Host охватывает
    вирусы с полными геномами, хранящимися в 1) NCBI/RefSeq и 2) GenBank, номера присоединения которых указаны в геномах
    EBI. Информация о хосте собирается из RefSeq, GenBank (в свободном текстовом формате), UniProt, ViralZone и вручную
    курируется с дополнительной информацией, полученной в результате опросов литературы. \cite{virus-host}
    
    \item GeneMarkS-2 - программное обеспечение для поиска генов внутри геномов. \cite{lomsad}
    
    \item NCBI Entrez Programming Utilities - программное обеспечение, набор из восьми серверных программ, которые
    обеспечивают стабильный интерфейс с системой запросов и баз данных Entrez в Национальном центре биотехнологической
    информации (NCBI) \cite{entrez}
    
    \item Biopython - набор свободно доступных инструментов для биологических вычислений, написанных на Python
    международной командой разработчиков. \cite{biopython}
    
    \item The MEME Suite - программное обеспечение для поиска новых, нераскрытых мотивов (повторяющихся паттернов
    фиксированной длины) в последовательностях. МЕМ разбивает паттерны переменной длины на два или более отдельных
    мотива. \cite{bailey}
    
    \item Python 3.8.5 - среда для запуска скриптов формата *.py \cite{python}
    
    \item Jupyter Notebook -веб-приложение с открытым исходным кодом, которое позволяет создавать и обмениваться
    документами, содержащими живой код, уравнения, визуализации и повествовательный текст. Использование включает в 
    себя:
    очистку и преобразование данных, численное моделирование, статистическое моделирование, визуализацию данных, 
    машинное
    обучение и многое другое. \cite{jupyter}
    
    \item Скрипты отсюда \(https://github.com/poolsar42/phages-and-hosts\) - подумать, что про них написать!
    \cite{github}
    
    \item \(Fasta\_merge.ipynb\) - скрипт для объединения регуляторных областей хозяина и его фагов в один файл.
    \item \(Hosts\_upstream\_regions.ipynb\) - скрипт для выделения регуляторных областей.
    \item \(Proteins\_formation.ipynb\) - скрипт для сбора белковых последовтельностей в один файл.
    \item \(Searching\_host\_genomes.ipynb\) - скрипт для выгрузки необходимых к работе геномов хостов.
    \item \(Unpacking.ipynb\) - скрипт для распаковки выгруженных геномов.
    \item \(VH\_tsv\_writer.ipynb\) - скрипт, выписывающий в файл ID хоста и ID прилежащих к нему фагов.
    \item \(Virus\_genes.ipynb\) - скрипт для выгрузки геномов фагов.
    \item \(missing\_phage\_genomes.ipynb\) - скрипт для поиска фагов, для которых GMS2 не находит гены.
    \item \(proteins\_convertation.ipynb \) - скрипт для преобразования \\ ДНК-последовательностей
                                в белковые.
    \item \(regex\_promoter.ipynb \) - скрипт для анализа полученных результатов
    \item \(run\_gms2.py\) - скрипт для поиска генов в геномах
    \item \(tsv\_write.ipynb\) - скрипт для выписывания информации о связи между хостами и фагами в один файл
\end{itemize}

\begin{center}
      \item \subsection{Получение общих мотивов}
      \item  \subsubsection{Протокол выгрузки геномов}
\end{center}
    \begin{itemize}
        \item С FTP сервера Virus-Hist DataBase выгрузить файл virushostdb.tsv - в нём содержится информация о связи 
        между фагами и их хостами.
    
        \item Запустить скрипт \(tsv\_write.ipynb\) - выписывает в удобном формате все взаимоотношения между фагами и 
        хостами.
    
      \item Запустить скрипт \(Virus\_genes.ipynb\) - для выгрузки геномов фагов.
        \item Запустить скрипт \(Searching\_host\_genomes.ipynb\) - для выгразки геномов хостов.
        \item Запустить скрипт \(Unpacking.ipynb\).

    \end{itemize}

    \begin{center}
    \item \subsubsection{Протокол получение вышестоящих регуляторных регионов}
    \end{center}
        \begin{itemize}
            \item Запустить скрипт \(run\_gms2.py\)
            \item Запустить скрипт \(Hosts\_upstream\_regions.ipynb\)
        \end{itemize}
    \begin{center}
    \item \subsubsection{Протокол подготовки данных к поиску мотивов}
    \end{center}
        \begin{itemize}
            \item Запустить скрипт \(Fasta\_merge.ipynb\)
            \item Запустить MEME Suite с получившимися файлами.
        \end{itemize}

\par{Нам нужно было посмотреть на сходтсва между вышестоящими регуляторными участками у хостов и фагов. Для этого нужно
было посмотреть на геномы хостов и фагов. Мы собрали геномы всех фагов, которые находятся в базе данных Virus-Host DB,
там для каждого фага также был указан или были указаны хосты, к которым он может прикрепиться. Их RefSeq ID. Мы всё
выгрузили с FTP сервера.}

\par{Далее мы выгрузили все геномы хостов с базы данных RefSeq NCBI. Когда у нас были геномы фагов и их хостов, нам было
нужно получить с этих геномов гены. Гены мы находили с помощью Gene-Marks 2. Что хорошо, Gene-Marks 2 также дал позицию
каждого гена в геноме. Каждому гену посчитали его транслирующийся белок, также с помощью Gene-Marks 2. После этого для
каждого гена нашли вышестоящие регуляторные области, длиной в 50 нуклеотидов, которые нужны для посадки 
РНК-полимераз(?).
Наша гипотеза как раз заключается в том, что эти вышестоящие области у фага и его хозяина для некоторых генов будут
совпадать по некоторым мотивам.}

\par{Мы поработали с последовательностями. Мы выгружали геномы всех фагов и хостов отдельными файлами. Мы предварительно
обработали файлы, достали из них вышестоящие регуляторные области длинной в 50 нуклеотов. Вышестоящие области хоста и
всех фагов, прилегающих к нему, мы закинули в один файл - в этом файле мы и искали общие мотивы.}

\par{Далее проверку на совпадения вышестоящих областей мы засунули все вышестоящие области хоста и его фагов в программу
MEME Suite. Как и ожидалось, мы получили множество профилей, которые были общими у фагов и их хоста. }

\par{Также MEME-Suite для каждого профиля мотива дала нам PFM (Position Frequency Matrix), поэтому у нас была 
возможность
сравнить сами мотивы между собой. У нас появилась гипотеза, что мотивы фагов, атакующих одного хоста, могут быть также
схожи и как-то ировать между собой. Мы решили посмотреть корреляцию 2D массивов - потому что для каждой позиции в
матрице частот у нас было 4 значения - на 4 возможных нуклеотида.}

\newpage
\begin{center}
    \item \section{Результаты и Обсуждение}
    \item \subsection{MEME-Suite Output}
\end{center}

\begin{itemize}
    \item Логотипы мотивов - Логотипы последовательностей представляют собой графическое представление информационного
    содержимого, хранящегося в системе множественного выравнивания последовательностей (MSA), и обеспечивают
    компактное и интуитивно понятное представление специфичного для положения нуклеотидного состава связывающих
    мотивов, активных сайтов и т.д. в биологических последовательностях. (Reference: Thomsen, M.C., \& Nielsen, M.
    2012. Nucleic Acids Res. 40(Web Server issue):W281-287).
   
    \item Матрица частот нуклеотидов - считается путем подсчета вхождений каждого нуклеотида в каждой позиции. Один 
    мотив встречается в последовательностях несколько раз и с наперёд заданной точностью мы считаем, что это подходящий 
    для нас мотив.
    
    \item PSSM (Position-Specific Scoring Matrices) - The position-specific scoring matrix corresponding to the motif
    is printed for use by database search programs such as MAST. This matrix is a log-odds matrix calculated by taking
    100 times the log (base 2) of the ratio p/f at each position in the motif where p is the probability of a
    particular letter at that position in the motif, and f is the background frequency of the letter (given in the
    command line summary section.) This is the same matrix that is used above in computing the p-values of the
    occurrences of the motif in the Occurrences of the Motif and Block Diagrams of Motif Occurrences sections. The
    scoring matrix is printed "sideways"\-\-columns correspond to the letters in the alphabet (in the same order as
    shown in the simplified motif) and rows corresponding to the positions of the motif, position one first. The
    scoring matrix is preceded by a line starting with "log-odds matrix:" and containing the length of the alphabet,
    width of the motif, number of characters in the training set, the scoring threshold (obsolete) and the motif
    E-value. \cite{memeres} 
    
    \item host\_phage\_results\_total.tsv - таблица, в которой содержатся данные о мотивах: последовательность, длина, 
    число сайтов, ic, llr, e\_value, название файла с PSSM, название файла с PFM, регулярное выражения для мотива и 
    причина остановки.
\end{itemize}


\begin{center}
\item \subsection{Обсуждения}
\end{center}

    \par{Прежде всего, мы уже знаем, что у бактерии и у хоста повторяются -35 и -10 промоторные элементы. Нам они не
    были интересны. Мы нашли возможные последовательности для -35 и для -10 участков \cite{-35,-10,-35-10,-35-10wiki}.
    Нашли и удалили их из нашего датасета с помощью регулярных выражений \cite{re}. }
    
    \par{Нам было интересно положение мотивов относительно начала транскрипции. Как уже было сказано, мы выделили 50 
    нуклеотидов - перед стартом транскрипции. Опять же, с помощью регулярных выражений, мы нашли положения мотивов в
    каждом 50-нуклеотидном регионе, в котором он встречается. Для каждого мотива мы посчитали среднее значение для его
    левого конца относительно начала транскрипции, для его правого конца и среднеквадратичную ошибку для обоих
    концов. Результаты записали в файл $ host\_phage\_results\_total\_updated.tsv $}
    
    \par{Дальше для каждого мотива мы нашли где контретно он находится и добавили в host\_phage\_results\_total.tsv
    отдельную колонку ``location''. В этой колонке присутствуют значения BOTH - если мотив был найден и у бактерии и у
    фага, либо ``PHAGE'', если мотив был найден только и фага или ``HOST'', если мотив был найден только у бактерии. К
    нашему удивлению некоторые мотивы не были найдены нигде - для таких мы в этой колонке выписали ``NOT FOUND''.}
    
    \par{Нам также стало интересно, коррелируют ли как-то между собой мотивы бактериофагов, которые атакуют одну 
    бактерию, а также есть ли корреляция между мотивами бакториофагов из одной таксономической категории. Для начала мы 
    хотели посчитать корреляцию Пирсона между PFM этих мотивов. Но PFM - двумерный массив, мы не обнаружили в известных 
    библиотеках Python подходящей функции для подсчётка корреляции между 2D-массивами. Мы написали собственный скрипт 
    для предварительного подсчёта такой корреляции $preliminary_correlation.py$, используя трюки отсюда 
    \cite{stackoverflow}. Результаты на данном этапе трудно интерпретировать, но можно заметить, что данные и вправду 
    как-то коррелируют.}
    
    \par{Также, как я уже написал выше, мы удивлись, когда обнаружили, что некоторые мотивы не были найдены ни в каком 
    из представленных вышестоящих регуляторных регионах. Но мы искали сами мотивы внутри них. Также во многих мотивах 
    можно заметить большое количество букв W, N, Y и ещё некоторых, которые не входят в состав привычных нам A,T,G и C 
    для нуклеотидов. Дело в том, что на одном месте в мотиве в разных последовательностях могут стоять разные 
    нуклеотиды. И для таких двойственных мест MEME-Suite использует специальный алфавит, в котором одна буква заменяет 
    все возможные основания на данной позиции \cite{memealphabet}. }
    
    \par{Но когда мы начали анализировать эти последовательности, мы заметили, что некоторые регулярные выражения и 
    двойственные буквы, которые должны им соответствовать, не совпадают с тем, что дано в алфавите \cite{memealphabet}. 
    Мы решили протестировать нашу выборку в ещё одно программе по поиску собственных мотивов. К сожаления, пока что мы 
    не нашли подходящей программы.
    \begin{itemize}
        \item sarus-2.0.2 ищет только известный мотив в нескольких последовательностях (а нам нужно в них самих находить
        новые). 
        \item Weeder подходит под наши критерии, но он ищет только мотивы длиной 6, 8 и 10 нуклеотидов - для нас это 
        маловато. мы же искали мотивы длиной от 10 до 50 нуклеотидов.
        \item Phyloscan, motifmatchr и много других пакетов для R всё ещё мне недоступны. Я с R впервые познакомился в 
        этом семестре, поэтому я не успел пока к нему приспособиться, чтоби использовать в дипломе какие-то пакеты для 
        R.
        \item MotifScan и много подобных ему ищут лишь мотивы из известных баз данных. Здесь мы столкнулись с той же 
        проблемой, что и при использовании sarus-2.0.2
    \end{itemize}
    }
    
    \par{Кратко описал результаты. Видно сейчас, что работа на будущее у нас есть: мы хотим написать свой скрипт для 
    корелляционого анализа 2D массивов и + проверить наши мотивы в другой прогаммы для уверенности.}
    
\newpage
\begin{center}
\item \section{Выводы}
\end{center}

\newpage
\begin{thebibliography}{9}
    \addcontentsline{toc}{section}{\refname}
    \bibitem{hinton1} {Stitt B, Hinton DM: Regulation of middle-mode transcription.In Molecular biology  of  
    bacteriophage  T4.Edited by: Karam JD, Drake J, Kreuzer KN, Mosig G, Hall D, Eiserling F, Black L, Spicer E, Kutter 
    E, Carlson K, Miller ES.Washington, D.C.: American Society for Microbiology; 1994:142-160}
    \bibitem{hinton2} {Brody E, Rabussay D, Hall D:Regulation of transcription of prereplicative genes. In Bacteriophage
    T4.Edited by: Mathews CK, Kutter EM, Mosig G,Berget PB. Washington, D. C.: American Society for 
    Microbiology;1983:174-183}
    \bibitem{hinton3} Hinton DM, Pande S, Wais N, Johnson XB, Vuthoori M, Makela A, Hook-Barnard I: Transcriptional 
    takeover by sigma appropriation: remodelling of the sigma70 subunit of Escherichia coli RNA polymerase by the 
    bacteriophage T4 activator MotA and co-activator AsiA.Microbiology2005,151:1729-1740.
    \bibitem{hinton4} Miller ES, Kutter E, Mosig G, Arisaka F, Kunisawa T, Ruger W: BacteriophageT4 genome. Microbiol  
    Mol  Biol  Rev2003,67:86-156
    \bibitem{hinton5} Wilkens K, Ruger W: Transcription from early promoters. In Molecular Biology  of  Bacteriophage  
    T4.Edited by: Karam JD, Drake JW, Kreuzer KN,Mosig G, Hall DH, Eiserling FA, Black LW, Spicer EK, Kutter E, Carlson 
    K,Miller ES. Washington, D. C.: American Society for Microbiology;1994:132-141.
    \bibitem{hinton6} Weisberg R, Hinton DM, Adhya S: Transcriptional Regulation in Bacteriophage. In Encyclopedia  of  
    Virology.3 edition. Edited by: Mahy BWJ,van Regenmortel MHV. Oxford: Elsevier; 2008:174-186, 174-186.
    \bibitem{hinton7} Hinton DM: Transcription from a bacteriophage T4 middle promoter using T4 motA protein and 
    phage-modified RNA polymerase.J  BiolChem 1991, 266:18034-18044
    \bibitem{hinton8} Schmidt RP, Kreuzer KN: Purified MotA protein binds the -30 region of a bacteriophage T4 
    middle-mode promoter and activates transcription invitro. J  Biol  Chem1992, 267:11399-11407
    \bibitem{hinton9} Stevens A: New small polypeptides associated with DNA-dependent RNA polymerase of Escherichia coli
    after infection with bacteriophage T4. Proc  Natl  Acad  Sci  USA1972, 69:603-607.
    \bibitem{hinton10} Stevens A:An inhibitor of host sigma-stimulated core enzyme activitythat purifies with 
    DNA-dependent RNA polymerase of E. coli followingT4 phage infection.Biochem  Biophys  Res  Commun1973,54:488-493.
    \bibitem{hinton11} Ouhammouch M, Orsini G, Brody EN: The asiA gene product of bacteriophage T4 is required for 
    middle mode RNA synthesis.J  Bacteriol1994, 176:3956-3965
    \bibitem{hinton12} Ouhammouch M, Adelman K, Harvey SR, Orsini G, Brody EN: BacteriophageT4 MotA and AsiA proteins 
    suffice to direct Escherichia coli RNA polymerase to initiate transcription at T4 middle promoters. Proc  NatlAcad  
    Sci  USA1995, 92:1451-1455.
    \bibitem{hinton13} Hinton DM, March-Amegadzie R, Gerber JS, Sharma M: Bacteriophage T4middle transcription system: 
    T4-modified RNA polymerase; AsiA, a sigma70 binding protein; and transcriptional activator MotA.Methods  
    Enzymol1996, 274:43-57
    \bibitem{virus-host} https://www.genome.jp/virushostdb/ 
    \bibitem{biopython} https://biopython.org/
    \bibitem{python} https://docs.python.org/3/
    \bibitem{jupyter} https://jupyter.org/
    \bibitem{github} https://github.com/poolsar42/phages-and-hosts
    \bibitem{entrez} https://www.ncbi.nlm.nih.gov/books/NBK25501/
    \bibitem{hinton} Hinton D.M. Transcriptional control in the prereplicative phase of T4 development // Virology
    journal. 2010. V. 7. P. 289. https://doi.org/10.1186/1743-422X-7-289
    \bibitem{puxty-evanx} Puxty, R.J., Evans, D.J., Millard, A.D. et al. Energy limitation of cyanophage development:
    implications for marine carbon cycling // ISME J. 2018, 12, 1273–1286. https://doi.org/10.1038/s41396-017-0043-3
    \bibitem{lomsad} Lomsadze, A., Gemayel, K., Tang, S., and Borodovsky, M. Modeling leaderless transcription and
    atypical genes results in more accurate gene prediction in prokaryotes // Genome research,  2018. 28(7), 1079–1089.
    https://doi.org/10.1101/gr.230615.117
    \bibitem{bailey} Bailey, T.L., Johnson, J., Grant, C.E., and Noble, W.S. The MEME Suite // Nucleic acids research,
    2015, 43(W1), W39–W49. https://doi.org/10.1093/nar/gkv416
    \bibitem{ryasik} Ryasik, A., Orlov, M., Zykova, E., Ermak, T., Sorokin, A. Bacterial promoter prediction: Selection
    of dynamic and static physical properties of DNA for reliable sequence classification // 2018
    https://www.worldscientific.com/doi/abs/10.1142/S0219720018400036
    \bibitem{genomics} Roger W. Hendrix // Bacteriophage genomics // Current Opinion in Microbiology October 2003, Pages
    506-511. https://doi.org/10.1016/j.mib.2003.09.004
    \bibitem{structure} Cesar O. Flores, Justin R. Meyer, Sergi Valverde et al. // Statistical structure of host–phage 
    interactions // PNAS July 12, 2011 108 (28) E288-E297; https://doi.org/10.1073/pnas.1101595108 
    \bibitem{networks} Joshua S.Weitz, Timothee Poisot, Justin R. Meyer et al. // Phage–bacteria infection networks // 
    Trends in Microbiology February 2013, Pages 82-91 https://doi.org/10.1016/j.tim.2012.11.003
    \bibitem{memeres} https://kodomo.fbb.msu.ru/~partyhard/term4/pr9/meme\_out\_oops/meme.html\#pssm1
    \bibitem{-35} Promotors Addgene https://www.addgene.org/mol-bio-reference/promoters/
    \bibitem{-35-10} Chang-Hui Shen // Gene Expression: Transcription of the Genetic Code // Diagnostic Molecular 
    Biology, 2019 https://doi.org/10.1016/B978-0-12-802823-0.00003-1
    \bibitem{-10} Andrey Feklistov,  Seth A. Darst // Structural Basis for Promoter $\sim$10 Element Recognition by the 
    Bacterial RNA Polymerase $\sigma$ Subunit // Cell Volume 147, Issue 6, 9 December 2011, Pages 1257-1269 
    https://doi.org/10.1016/j.cell.2011.10.041
    \bibitem{-35-10wiki} https://en.wikipedia.org/wiki/Promoter\_(genetics)\#Bacterial
    \bibitem{re} https://docs.python.org/3/library/re.html\#module-re
    \bibitem{stackoverflow} https://stackoverflow.com/questions/30143417/computing-the-correlation-coefficient-between-t
    wo-multi-dimensional-arrays
    \bibitem{memealphabet} https://meme-suite.org/meme/doc/alphabet-format.html\#standard\_DNA
    \bibitem{meme} Bailey TL, Elkan C. Fitting a mixture model by expectation maximization to discover motifs in 
    biopolymers. Proc Int Conf Intell Syst Mol Biol. 1994;2:28-36. PMID: 7584402
    \bibitem{phagewikiru} https://ru.wikipedia.org/wiki/Бактериофаги
    \bibitem{phagewikieng} https://en.wikipedia.org/wiki/Bacteriophage
    \bibitem{phageapps} Monk, A., Rees, C., Barrow, P., Hagens, S. and Harper, D. (2010), Bacteriophage applications: 
    where are we now? Letters in Applied Microbiology, 51: 363-369. https://doi.org/10.1111/j.1472-765X.2010.02916.x
    \bibitem{phagetreat} David R. Harper, Benjamin H Burrowes, Elizabeth M. Kutter (15 August 2014) Bacteriophage: 
    Therapeutic Uses, Letters in Applied Microbiology.  https://doi.org/10.1002/9780470015902.a0020000.pub2
    \bibitem{phagemed} https://onlinelibrary.wiley.com/doi/10.1111/j.1749-4486.2009.01973.x
    \bibitem{phagetherapy} Sulakvelidze, A., Alavidze, Z., \& Morris, J. G., Jr (2001). Bacteriophage therapy. 
    Antimicrobial agents and chemotherapy, 45(3), 649–659. https://doi.org/10.1128/AAC.45.3.649-659.2001
    \bibitem{advdisphage} https://sitn.hms.harvard.edu/flash/2018/bacteriophage-solution-antibiotics-problem/
    \bibitem{phagesalm} J.P. Higgins, S.E. Higgins, K.L. Guenther, W. Huff, A.M. Donoghue, D.J. Donoghue, B.M. Hargis, 
    Use of a specific bacteriophage treatment to reduce Salmonella in poultry products12, Poultry Science, Volume 84, 
    Issue 7, 2005, Pages 1141-1145, ISSN 0032-5791, https://doi.org/10.1093/ps/84.7.1141
    \bibitem{phagetrans} Bing Liu, Andrey Shadrin, Carol Sheppard, Vladimir Mekler, Yingqi Xu, Konstantin Severinov, 
    Steve Matthews, Sivaramesh Wigneshweraraj, A bacteriophage transcription regulator inhibits bacterial transcription 
    initiation by $\sigma$-factor displacement, Nucleic Acids Research, Volume 42, Issue 7, 1 April 2014, Pages 
    4294–4305, 
    https://doi.org/10.1093/nar/gku080 
    \bibitem{17} Paget MS, Helmann JD: The sigma70 family of sigma factors. Genome Biol 2003, 4: 203. 
    10.1186/gb-2003-4-1-203
    \bibitem{18} Gruber TM, Gross CA: Multiple sigma subunits and the partitioning of bacterial transcription space. 
    Annu Rev Microbiol 2003, 57: 441-466. 10.1146/annurev.micro.57.030502.090913
    \bibitem{19}  Campbell EA, Westblade LF, Darst SA: Regulation of bacterial RNA polymerase sigma factor activity: a 
    structural perspective. Curr Opin Microbiol 2008, 11: 121-127. 10.1016/j.mib.2008.02.016
    \bibitem{20} Hook-Barnard IG, Hinton DM: Transcription Initiation by Mix and Match Elements: Flexibility for 
    Polymerase Binding to Bacterial Promoters. Gene Regulation and Systems Biology 2007. 
    http://la-press.com/article.php?article\_id=481:275-293
    \bibitem{4} Miller ES, Kutter E, Mosig G, Arisaka F, Kunisawa T, Ruger W: Bacteriophage T4 genome. Microbiol Mol 
    Biol Rev 2003, 67: 86-156. 10.1128/MMBR.67.1.86-156.2003
    \bibitem{150} Estrem ST, Ross W, Gaal T, Chen ZW, Niu W, Ebright RH, Gourse RL: Bacterial promoter architecture: 
    subsite structure of UP elements and interactions with the carboxy-terminal domain of the RNA polymerase alpha 
    subunit. Genes Dev 1999, 13: 2134-2147. 10.1101/gad.13.16.2134
    \bibitem{91} Stoskiene G, Truncaite L, Zajanckauskaite A, Nivinskas R: Middle promoters constitute the most abundant
    and diverse class of promoters in bacteriophage T4. Mol Microbiol 2007, 64: 421-434. 
    10.1111/j.1365-2958.2007.05659.x
    \bibitem{5} Wilkens K, Ruger W: Transcription from early promoters. In Molecular Biology of Bacteriophage T4. Edited
    by: Karam JD, Drake JW, Kreuzer KN, Mosig G, Hall DH, Eiserling FA, Black LW, Spicer EK, Kutter E, Carlson K, Miller
    ES. Washington, D. C.: American Society for Microbiology; 1994:132-141.
    \bibitem{39} Wilkens K, Ruger W: Characterization of bacteriophage T4 early promoters in vivo with a new promoter 
    probe vector. Plasmid 1996, 35: 108-120. 10.1006/plas.1996.0013
    \bibitem{2_} Flombaum P, Gallegos JL, Gordillo RA, Rincón, J, Zabala LL, Jiao N, et al. Present and future global 
    distributions of the marine cyanobacteria Prochlorococcus and Synechococcus. Proc Natl Acad Sci USA. 2013;110: 
    9824–9829.
    \bibitem{3_} Zwirglmaier K, Jardillier L, Ostrowski M, Mazard S, Garczarek L, Vaulot D, et al. Global phylogeography
    of marine Synechococcus and Prochlorococcus reveals a distinct partitioning of lineages among oceanic biomes.
    Environ Microbiol. 2008;10:147–161.
    \bibitem{4_} Bouman HA, Ulloa O, Scanlan DJ, Zwirglmaier K, Li WKW, Platt T, et al. Oceanographic basis of the 
    global surface distribution of Prochlorococcus ecotypes. Science. 2006;312:918–921.
    \bibitem{5_} Scanlan DJ, Ostrowski M, Mazard S, Dufresne A, Garczarek L, Hess WR, et al. Ecological genomics of 
    marine picocyanobacteria. Microbiol Mol Biol Rev. 2009;73:249–299.
    \bibitem{6_} Biller SJ, Berube PM, Lindell D, Chisholm SW. Prochlorococcus: the structure and function of collective
    diversity. Nat Rev Microbiol. 2015;13:13–27.
    \bibitem{7_} Lindell D, Jaffe JD, Johnson ZI, Church GM, Chisholm SW. Photosynthesis genes in marine viruses yield 
    proteins during host infection. Nature. 2005;438:86–89.
    \bibitem{8_} Mann NH, Cook A, Millard AD, Bailey S, Clokie M. Bacterial photosynthesis genes in a virus. Nature. 
    2003;424:741–742.
    \bibitem{9_} Puxty RJ, Millard AD, Evans DJ, Scanlan DJ. Viruses inhibit CO2 fixation in the most abundant 
    phototrophs on Earth. Curr Biol. 2016;26:1585–1589.
    \bibitem{18_} Garczarek L, Dufresne A, Blot N, Cockshutt AM, Peyrat A, Campbell DA, et al. Function and evolution of
    the psbA gene family in marine Synechococcus: Synechococcus sp. WH7803 as a case study. ISME J. 2008;2:937–953.
    \bibitem{23_} Mulo P, Sakurai I, Aro EM. Strategies for psbA gene expression in cyanobacteria, green algae and 
    higher plants: From transcription to PSII repair. Biochim Biophys Acta Bioenerg. 2012;1817:247–257.
    \bibitem{24_} Imamura S, Yoshihara S, Nakano S, Shiozaki N, Yamada A, Tanaka K, et al. Purification, 
    characterization, and gene expression of all sigma factors of RNA polymerase in a cyanobacterium. J Mol Biol. 
    2003;325:857–872.
    \bibitem{25_} Imamura S, Asayama M, Shirai M. In vitro transcription analysis by reconstituted cyanobacterial RNA 
    polymerase: roles of group 1 and 2 sigma factors and a core subunit, RpoC2. Genes Cells. 2004;9:1175–1187.
    \bibitem{31_} Miller ES, Kutter E, Mosiq G, Arisaka F, Kunisawa T, Rüger W. Bacteriophage T4 genome. Microbiol Mol 
    Biol Rev. 2003;67:86–156.
    \bibitem{_1}   Altman, R, Brutlag, D, Karp, P, Lathrop, R, \& Searls, D. Proceedings: Second international conference on intelligent systems for molecular biology. United States. 
\end{thebibliography}

\end{document}
