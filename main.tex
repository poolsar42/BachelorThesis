\documentclass[a4paper,12pt]{article}

\usepackage{cmap}		
\usepackage[utf8]{inputenc}			
\usepackage[english,russian]{babel}
\usepackage{framed}
\usepackage{hyperref}
\usepackage{amsmath}
\usepackage{graphicx}
\usepackage[colorinlistoftodos]{todonotes}
\usepackage{wrapfig}
\usepackage{lipsum}
\usepackage{listings}
\usepackage{color}
\usepackage{indentfirst}
\usepackage{times}
\usepackage{textcomp}


\definecolor{mygray}{rgb}{0.4,0.4,0.4}
\definecolor{mygreen}{rgb}{0,0.8,0.6}
\definecolor{myorange}{rgb}{1.0,0.4,0}

\lstdefinestyle{customc}{
  belowcaptionskip=1\baselineskip,
  breaklines=true,
  frame=L,
  xleftmargin=\parindent,
  language=C,
  showstringspaces=false,
  basicstyle=\footnotesize\ttfamily,
  keywordstyle=\bfseries\color{green!40!black},
  commentstyle=\itshape\color{purple!40!black},
  identifierstyle=\color{blue},
  stringstyle=\color{orange},
  numbers=left,
  numbersep=13pt,
  numberstyle=\small\color{mygray},
}
\lstset{escapechar=@,style=customc}
\linespread{1.6}
\newcommand{\HRule}{\rule{\linewidth}{0.5mm}}

\begin{document}

\begin{titlepage}
\begin{center}

\textsc{\Large Московский Физико-Технический Институт}\\
\textsc{\large (Национальный Исследовательский Университет)}\\[1.5cm]

% Upper part of the page. The '~' is needed because \\
% only works if a paragraph has started.
\includegraphics[width=0.25\textwidth]{img/logo.png}~\\[1cm]

\textsc{\Large Бакалаврская работа}\\[0.5cm]

% Title
\HRule \\[0.4cm]
{ \LARGE \bfseries Влияние вышестоящих регуляторных регионов и промоторов на связь между бактериофагами и их хозяевами \\[0.4cm] }

\HRule \\[1.5cm]

% Author and supervisor
\noindent
\begin{minipage}{0.4\textwidth}
\begin{flushleft} \large
\emph{Студент:}\\
Шарафутдинов Эмиль
\end{flushleft}
\end{minipage}%
\begin{minipage}{0.4\textwidth}
\begin{flushright} \large
\emph{Преподаватель:} \\
 Родригес Валера Франсиско Эдуардо
\end{flushright}
\end{minipage}

\vfill

% Bottom of the page
{\large 24 апреля 2021 г.}

\end{center}
\end{titlepage}

\newpage
\section{Аннотация}

Бактериофаги (фаги) - это вирусы, которые избирательно поражают бактериальные клетки. Такая специфика обусловлена
зависимостью фага от клеточного механизма хозяина, который нужен фагу для репликации и биосинтеза вирусных компонентов.
Пример такой зависимости  можно привести для фага Т4. Во время транскрипции, ранняя конкуренция за РНК-полимеразу хозяина
позволяет вирусу полностью захватить механизм репликации путем сигма-присвоения(sigma appropriation) [1]. Недавно
показали[2], что фаги, которые заражают цианобактерии, кодируют ген фотосинтеза. Причём транскрипция этого гена реагирует
на те же условия окружающей среды, что и ген, который кодируется у хозяина. Из этого можно предположить, что существует
связь между регуляторными элементами фага и регуляторными элементами их хозяина. Это поможет лучше предсказывать
взаимоотношения фаг-хозяин.

\newpage
\tableofcontents



\newpage
\section{Введение}

    В одной статье были собраны результаты разных исследований, касающихся взаимоотношений фага T4 и его хоста – E. Coli.
    В этих работах было показано, как ведёт себя фаг, который внедряется в бактерию.
	У фага Т4 нет собственных РНК-полимераз, поэтому он использует РНК-полимеразы хоста для транскрипции. Но почему РНК
	полимераза хоста должна сразу транскрибировать именно с ДНК Т4? Не должна. Но всё же обнаружили, что некоторая часть
	ДНК фага транскрибируется РНК-полимеразой хоста сразу. Эту часть транскрипции назвали ранней транскрипцией фага. На
	этом слайде можно видеть, почему эта ранняя транскрипция просходит. Здесь обозначены разные мотивы промоторов ДНК
	фага (посередине – ранняя, снизу - промежуточная) и хоста (сверху). Эти мотивы, которые находятся в промоторе,
	называются вышестоящими регуляторными элементами. Как видно, в промоторе ранней транскрипции регуляторные элементы
	между фагом и хостом совпадают, и выше -35 участка, что позволяет РНК полимеразе цепляться за ДНК фага также хорошо,
	как за ДНК хоста.  Всё, что до мотива TGn совпадает у всех промоторов. MotA Box в промоторе промежуточной
	транскрипции нужен для зацепки белка, который способствует промежуточной транскрипции. Вкратце, что там просходит: в
	этот момент с РНК, транскрибировавшейся от фага (в ранней транскрипции), транслируются белки, которые модифицируют
	РНК-полимеразы хозяина таким образом, что далее начинают транскрибироваться только ДНК фага. В том числе в начальной
	транскрипции ещё образуется белок MotA, который позволяет ДНК прикрепиться к модифицированной РНК-полимеразе. После
	чего с этой ДНК начинает транскрибироваться РНК. Этот этап уже называют промежуточной транскрипцией. Этот процесс
	присвоения РНК-полимеразы хозяина называют сигма присвоением. Ещё раз. Проиcходит это так:

	В самом начале транслируются белки, которые глушат РНК-полимеразу
	Вместе с этим транслируются белки, которые позволяют при заглушенной полимиразе транскрибироваться ДНК фага на этой
	полимеразе, они садятся на ДНК фага
	Таким образом фаг полностью захватывает РНК-полимеразу хозяина
	Рассмотри ещё один интересный случай для нас. Ещё в одной статье авторы смотрели, как ведут себя транкриптомы фагов и
	их хозяев (цианобактерий в этом случае) в зависимости от интенстивности света, которым они светили. При маленькой интенсивности транскриптомы делились в среднем поровну, но при бОльшей интенсивности среднее число РНК фагов было
	намного больше, чем у цианобактерий. Авторы решили посмотреть, что это за участок РНК, который так сильно возрастает
	у фага при сильном свете. Они обнаружили, что это участок РНК, ответственный за белок psbA, который оказался общим у
	фага и его хозяина. Авторы предположили, что это вполне возможно, если фаги позаимствовали гены у хозяев в ходе
	рекомбинации.

\newpage
\section{Цель и задачи} \label{sec:code}

Эти статьи – хорошие примеры, но они не единственные в своём роде. Нам стали интересны такие события, и в нашей работе
мы предположили, что такие схожие вышестоящие регуляторные участки в ДНК фагов и их хостов могут встречаться не только у
цианобактерий и их фагов и T4 с E. Coli, а у всех фагов и их хостов.
В том числе, мы посчитали, что у многих фагов и их хозяев могут встречаться общие белки. Таким образом, если мы сможем
установить взаимоотношения фагов и их хозяинов, мы в будущем сможем прогнозировать отношения фагов и каких-либо хостов.

\newpage
\section{Обзор литературы} \label{sec:math}


\newpage
\section{Материалы и Методы} \label{sec:code}
\par{}
Нам нужно было посмотреть на сходтсва между вышестоящими регуляторными участками у хостов и фагов. Для этого нужно было
посмотреть на геномы хостов и фагов. Мы собрали геномы всех фагов, которые находятся в базе данных Virus-Host DB, там для
каждого фага также был указан или были указаны хосты, к которым он может прикрепиться. Их RefSeq ID. Мы всё выгрузили с
FTP сервера.
\par{}
Далее мы выгрузили все геномы хостов с базы данных RefSeq NCBI. Когда у нас были геномы фагов и их хостов, нам было нужно
получить с этих геномов гены. Гены мы находили с помощью Gene-Marks 2. Что хорошо, Gene-Marks 2 также дал позицию каждого
гена в геноме. Каждому гену посчитали его транслирующийся белок, также с помощью Gene-Marks 2. После этого для каждого
гена нашли вышестоящие регуляторные области, длиной в 50 нуклеотидов, которые нужны для посадки РНК-полимераз(?). Наша
гипотеза как раз заключается в том, что эти вышестоящие области у фага и его хозяина для некоторых генов будут совпадать
по некоторым мотивам.
\par{}
Мы поработали с последовательностями. Мы выгружали геномы всех фагов и хостов отдельными файлами. Мы предварительно
обработали файлы, достали из них вышестоящие регуляторные области длинной в 50 нуклеотов. Вышестоящие области хоста и
всех фагов, прилегающих к нему, мы закинули в один файл - в этом файле мы и искали общие мотивы.
\par{}
Далее проверку на совпадения вышестоящих областей мы засунули все вышестоящие области хоста и его фагов в программу MEME
Suite. Как и ожидалось, мы получили множество профилей, которые были общими у фагов и их хоста. 
\par{}
Также MEME-Suite для каждого профиля мотива дала нам PFM (Position Frequency Matrix), поэтому у нас была возможность
сравнить сами мотивы между собой. У нас появилась гипотеза, что мотивы фагов, атакующих одного хоста, могут быть также
схожи и как-то коррелировать между собой. Мы решили посмотреть корреляцию 2D массивов - потому что для каждой позиции в
матрице частот у нас было 4 значения - на 4 возможных нуклеотида.

\newpage
\section{Результаты и Обсуждение}

\newpage
\section{Выводы}


\newpage
\begin{thebibliography}{9}
    \addcontentsline{toc}{section}{\refname}
    \bibitem{hinton} Hinton D.M. Transcriptional control in the prereplicative phase of T4 development // Virology journal. 2010. V. 7. P. 289. https://doi.org/10.1186/1743-422X-7-289
    \bibitem{puxty-evanx} Puxty, R.J., Evans, D.J., Millard, A.D. et al. Energy limitation of cyanophage development: implications for marine carbon cycling // ISME J. 2018, 12, 1273–1286. https://doi.org/10.1038/s41396-017-0043-3
    \bibitem{lomsad} Lomsadze, A., Gemayel, K., Tang, S., and Borodovsky, M. Modeling leaderless transcription and atypical genes results in more accurate gene prediction in prokaryotes // Genome research,  2018. 28(7), 1079–1089. https://doi.org/10.1101/gr.230615.117
    \bibitem{bailey} Bailey, T.L., Johnson, J., Grant, C.E., and Noble, W.S. The MEME Suite // Nucleic acids research, 2015, 43(W1), W39–W49. https://doi.org/10.1093/nar/gkv416
    \bibitem{ryasik} Ryasik, A., Orlov, M., Zykova, E., Ermak, T., Sorokin, A. Bacterial promoter prediction: Selection of dynamic and static physical properties of DNA for reliable sequence classification // 2018 https://www.worldscientific.com/doi/abs/10.1142/S0219720018400036
    
\end{thebibliography}

\end{document}

\usepackage[english,russian]{babel}